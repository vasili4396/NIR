\section{Постановка задачи}

В наши дни ключевые носители, такие как смарт-карты и токены, распространены повсеместно и считаются безопасными, защищенными от взлома и
заслуживающими доверия устройствами. Они используются для выполнения конфиденциальных операций,
таких как идентификация и аутентификация пользователей, а также для хранения и обработки конфиденциальных данных.
Такие операции подразумевают взаимодействие между ключевым носителем и сторонними недоверенными системами.

Согласно \cite[c.~159]{dit} защищенный ключевой носитель должен обладать несколькими свойствами, одно из которых заключается в обеспечении
возможности защищенного хранения криптографических ключей с применением интерфейсов работы со смарт-картой (CCID или PKCS\#11).

CCID, как стандарт интерфейса USB, является одним из наиболее распространенных на сегодняшний день\cite{iccid}.
Вендоры смарт-карт и токенов предоставляют разработчикам возможность взаимодействовать со своими устройствами посредсвом проприетарных
команд. Тем не менее, часто документация находится в закрытом доступе, либо ее вообще может не быть, потому что разработчик
не позаботился об этом. В таких случаях, работа с аппаратными CCID-токенам может осуществляться лишь методом научного тыка. Либо можно подойти
к задаче более систематически и попробовать разобраться, что происходит в устройстве и какие более низкоуровневые команды вызываются при
определенных действиях разработчика. К тому же, такой анализ позволит понять, насколько надежно спроектирована и реализована система
с точки зрения безопасности.

Обследование проводится в рамках предпроектных работ по теме «Разработка стенда для анализа взаимодействия с аппаратными CCID-токенами».

Заказчиком работ является кафедра защиты информации.

Исполнителем работ является студент кафедры защиты информации 519 группы
ФРТК МФТИ Иванов Василий Павлович.

\textbf{Объектом обследования} является процесс взаимодействия с аппаратными CCID-токенов.

\textbf{Целью обследования} является формирование требований для реализации
стенда, позволяющего провести анализ взаимодействия с аппаратными CCID-токенами.

Результаты проведенной работы отражены в настоящем отчете, который имеет
следующую структуру:
\begin{itemize}
  \item описание предметной области – Раздел 2;
  \item выбор и обоснование критериев качества – Раздел 3;
  \item анализ аналогов и прототипов – Раздел 4;
  \item перечень задач, решаемых в процессе разработки – Раздел 5;
  \item проект технического задания – Приложение 1.
\end{itemize}

\clearpage