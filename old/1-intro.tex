\section{Введение}
Одной из задач средств доверенной загрузки является контроль подключенной аппаратуры. ПАК «Аккорд-АМДЗ» позволяет выполнять контроль
целостности нескольких видов оборудования: процессора, BIOS, ОЗУ и других, в том числе и USB устройств. Администратор программно-аппаратного
комплекса имеет возможность конфигурировать список разрешенных для подключения USB устройств, сохраняя данную конфигурацию в
памяти контроллера \cite{admin_amdz}. Программное обеспечение «Аккорда» должно уметь не только идентифицировать подключенные USB устройства, но и
понимать их физическое расположение на шине.
\par
В текущей реализации комплекс «Аккорд-АМДЗ» работает в окружении операционной системы семейства Linux и использует утилиты своей ОС для
идентификации подключенных устройств. Но с появлением ЭВМ на базе UEFI BIOS, стала возможной реализация СДЗ уровня BIOS в виде EFI-приложений.
Такое решение позволило бы создать единую базу данных, которая будет хранить составленную администратором конфигурацию всего комплекса, и при этом
использоваться в разных окружениях.
\par
\textbf{Целью} данной работы является определение области применения существующего способа идентификации USB устройств в СДЗ «Аккорд-АМДЗ».
\par
Для достижения поставленной цели, необходимо решить следующие \textbf{задачи}:
\begin{itemize}
  \item Описать существующий механизм идентификации USB устройств в «Аккорд-АМДЗ»
  \item Составить список потенциальных ограничений способа идентифкации USB устройств
  \item Провести эксперимент по проверке наличия данных ограничений
  \item Проанализировав проведённый эксперимент, сделать предположение об ограничениях способа идентифкации USB устройств
\end{itemize}
Поставленные задачи будут решаться с помощью анализа и дальнейшего синтеза научной и учебной литературы, проведения испытаний на
экспериментальном стенде вычислительных систем.
\par
В конце работы предполагается получить список ограничений и описание области применения для текущего способа идентификации USB устройств в
устройствах «Аккорд-АМДЗ».
\clearpage