\section{Существующий механизм идентификации USB устройств}
\subsection{Общие сведения}
Универсальная последовательная шина (USB) используется для подключения хоста, такого как ПК или рабочая станция, к ряду периферийных устройств.
Сама шина USB состоит из следующих элементов:
\iffalse
\begin{itemize}
  \item Хост-контроллер — это главный контроллер, который управляет работой
  всех подключенных к шине устройств. Допускается наличие только одного хост-контроллера на шине USB.
  \item Хаб(концентратор) - устройство, которое обеспечивает дополнительные
  порты на шине. Хаб распознает подключение и отключение устройств,
  также может управлять подачей питания на порты. Корневым хабом
  называется устройство, которое напрямую взаимодействует с хост-контроллером.
  \item Порт — точка подключения периферийных USB устройств
\end{itemize}
Протокол USB использует древовидную структуру с корневым хабом в качестве корня этого дерева, концентраторами(хабами) в качестве
внутренних узлов и периферийными устройствами в качестве листьев дерева \cite{usb_tree}.
\begin{figure}[H]
  \includegraphics[scale=0.5]{images/tree.png}
  \caption{Дерево USB устройств}
\end{figure}
Протокол USB предусматривает наличие у каждого USB устройства нескольких иерархических дескрипторов, которые описывают информацию
для хоста об устройстве - такую, как, например, что это за устройство, какую версию USB поддерживает, тип устройства
(например, хаб или USB флэш-накопитель или клавиатура), количество конечных точек \cite{descriptors}.
Наиболее общими ялвяются следующие дескрипторы:
\begin{itemize}
  \item Дескриптор устройства
  \item Дескриптор конфигурации
  \item Дескриптор интерфейса
  \item Дескриптор конечной точки
\end{itemize}
Любое USB устройство может иметь лишь один дескриптор устройства (device descriptor).
\par
В момент запуска операционной системы, происходит инициализация всех устройств, подключенных к шине PCI. Первым инициализируется хост-контроллер,
который является "мостом" между PCI и USB шинами. Далее по цепочке инициализируются все хабы и устройства, подключенные к шине USB.
Весь процесс инициализации устройств логируется системой в /var/log/dmesg,
и доступен для чтения через стандартную утилиту ядра Linux - dmesg (приложение, рис. \ref{img:dmesg}).
\par
В ядре Linux заложена поддержка разных версий протокола USB: usb 1.0, usb 2.0 и usb 3.0. За общение с устройствами каждой из версий
отвечает соответствующий драйвер: oHCI - для usb 1.0, eHCI - для usb 2.0 и xHCI - для usb 3.0 \cite {kernel_echi}. Эти драйвера
используются для обработки разных типов дескрипторов после получения информации от дескриптора устройства.
\par
После успешного обнаружения хаба/порта, ядро Linux нумерует его для представления дерева подключенных
хабов и устройств в удобочитаемом для человека виде. За нумерацию отвечает функция usb\_alloc\_dev из drivers/usb/core/usb.c \cite{enumeration}.
После этого в директории /sys/bus/usb/devices/ создаются ссылки, перейдя по которым можно получить подробную информацию (которая
берется из дескриптора устройства) о каждом USB устройстве, подключенном к системе (рис. \ref{img:sysbus}).
Имена, начинающиеся с «usb», относятся к USB-контроллерам. Номер, следующий после «usb» - это номер хаба на шине USB.
В примере, включенном в приложение, на хостовой машине имеется 9 usb-контроллеров.
Имя ссылки формируется по следующему шаблону: номер шины-порт.порт.порт \cite{linux_usb}.
Интерфейсы обозначаются суффиксами, имеющими вид: :конфигурация.интерфейс.
Посмотреть список подключенных устройств в древовидном представлении можно с помощью утилиты lsusb, включающейся во многие
дистрибутивы Linux.
\clearpage
\subsection{Идентификация в "Аккорд-АМДЗ"}
Как уже было сказано, администратор программно-аппаратного комплекса "Аккорд-АМДЗ" может конфигурировать т.н. whitelist USB устройств,
подключение которых при аутентификации пользователя, не вызывет ошибки контроля целостности. При добавлении USB устройства в список разрешенных
устройств, администратор может не только указать само устройство, но и порт, в который оно должен быть подключено. Устройство идентифицируется
по VID(Vendor ID) и PID(Product ID), которые предоставляет дескриптор устройства. Идентификация устройства никак не зависит от окружения,
в котором происходит, в то время как физическое расположение на шине предоставляет система, в которой происходит аутентификация пользователя.
Поскольку программная часть комплекса "Аккорд-АМДЗ" построена поверх ядра Linux, то номер порта при конфигурировании whitelist'а
USB устройств берется из системы и имеет вид, описанный выше.

\clearpage
\section{Потенциальные ограничения существующего способа идентификации USB устройств}
Теперь, описав общую схему определения подключенных USB устройств, можно сформулировать потенциальные ограничения для существующего
способа идентификации.
\par
Дерево USB устройств, упомянутое ранее, является лишь поддеревом всех аппаратных устройств, распознанных операционной системой. Дерево устройств
(device tree) представляет собой структуру данных, описывающую аппаратные компоненты конкретной машины, чтобы ядро операционной системы
могло использовать и управлять этими компонентами, включающими в себя: процессор, память, шины и периферийные устройства \cite{devtree}.
Многие крупные компании приняли участие в разработке платформонезависимого стандарта, в котором дано полное техническое описание
формата данных для дерева устройств и лучшие практики. Разработчики ядра Linux отмечают, что они придерживаются этого стандарта. Тем не менее,
в самой спецификации ничего не сказано о способах или рекомендациях по идентификации хабов и портов на шине USB в момент загрузки
ядра операционной системы \cite{usb_spec}.
Отсюда возникает предположение, что физическое расположение USB устройств на шине может зависеть от окружения, в котором происходит идентификация.
Это утверждение будет проверено в следующей главе.
\par
Стоит также отметить следующий факт: в документации к ядру Linux сказано, что идентификатор устройства на шине USB должен
оставаться постоянным, отражая физическое расположение на шине даже после перезагрузки. Но не гарантируется, что идентификатор
останется прежним при изменении аппаратной конфигурации хостовой машины, например, при удалении или добавлении новых контроллеров.

\clearpage
\section{Идентификация USB устройств в разных окружениях}
Исследование, проведенное в процессе написания этой работы, предполагает определить, как идентифицируются различные устройства на шине USB.
Считывание конфигурации дерева USB устройств осуществлялось на одной физической машине, на которую было установлено несколько разных операционных
систем семейства Linux и Windows, а также в UEFI Shell. Аппаратная конфигурация физической машины не менялась. Во время эксперимента считывалось
физическое расположение одного и того же USB флэш-накопителя на 2 разных портах USB 2.0 и 1 USB 3.0.
\subsection{ОС семейста Linux}
Эксперимент проводился на двух наиболее популярных дистрибутивах Linux: Ubuntu 18.04 и Centos 7.
Слева - CentOS 7, справа - Ubuntu 18.04.
\begin{figure}[H]
  \includegraphics[width=0.4\textwidth, height=4.5cm]{images/centos-left.png}
  \includegraphics[width=0.575\textwidth, height=4.5cm]{images/ubuntu-left.png}
  \caption{Идентификация устройства и нумерация хабов при использовании порта USB 3.0}
\end{figure}
\begin{figure}[H]
  \includegraphics[width=0.4\textwidth, height=4.5cm]{images/centos-middle.png}
  \includegraphics[width=0.575\textwidth, height=4.5cm]{images/ubuntu-middle.png}
  \caption{Идентификация устройства и нумерация хабов при использовании порта USB 2.0}
\end{figure}
\begin{figure}[H]
  \includegraphics[width=0.4\textwidth, height=4.5cm]{images/centos-right.png}
  \includegraphics[width=0.575\textwidth, height=4.5cm]{images/ubuntu-right.png}
  \caption{Идентификация устройства и нумерация хабов при использовании порта USB 2.0}
\end{figure}
\par
Идентификация еще одного USB устройства класса Mass Storage в случае с CentOS связана лишь с тем, что операционная система была запущена
с live-образа, находящегося на флэш-накопителе. На чистоту эксперимента это никак не влияет.
\par
Результат эксперимента не удивителен. Оба проверенных дистрибутива используют одно и то же ядро Linux, и, соответственно, один и тот же код,
для нумерации и идентификации USB устройств. Отсюда можно сделать вывод, что одно и то же USB устройство, подключенное в один и тот же порт, будет
идентифицировано одинаково в разных дистрибутивах Linux.
\clearpage
\subsection{ОС семейста Windows}
В этом случае эксперимент проводился для двух разных версий Windows: 7 и 10. Выбор был сделан на основании того, что
это две самые популярные версии семейста Windows, а Windows XP и более ранние официально не поддерживаются.
\par
Слева - Windows 7, справа - Windows 10.
\begin{figure}[H]
  \includegraphics[width=0.5\textwidth, height=10.5cm]{images/7-left.png}
  \includegraphics[width=0.5\textwidth, height=10.5cm]{images/10-left.png}
  \caption{Физическое расположение на шине при использовании порта USB 3.0}
\end{figure}
\begin{figure}[H]
  \includegraphics[width=0.5\textwidth, height=10.5cm]{images/7-middle.png}
  \includegraphics[width=0.5\textwidth, height=10.5cm]{images/10-middle.png}
  \caption{Физическое расположение на шине при использовании порта USB 2.0}
\end{figure}
\begin{figure}[H]
  \includegraphics[width=0.5\textwidth, height=10.5cm]{images/7-right.png}
  \includegraphics[width=0.5\textwidth, height=10.5cm]{images/10-right.png}
  \caption{Физическое расположение на шине при использовании порта USB 2.0}
\end{figure}
Посмотреть физическое расположение на шине можно либо через диспетчер устройств, либо используя команду
Get-ItemProperty 'HKLM:\textbackslash System\textbackslash CurrentControlSet\textbackslash Enum\textbackslash USB\textbackslash
идентификатор usb устройства'.
По результатам эксперимента можно понять, что в разных версиях Windows, идентификация физического расположения на шине происходит по-разному и
отличается от расположения, которое наблюдалось в ОС семейства Linux.
\par
Ядро ОС Windows использует похожую организацию дерева устройств, что и Linux. В документации Windows сказано, что за нумерацию устройств
в момент загрузки ядра ОС отвечает
драйвер ACPI.sys \cite{acpi}. Стандарт(спецификация) ACPI определяет способы программного управления электропитанием компонентов
машины с помощью встроенных средств ОС. Тем не менее в самой спецификации ACPI не было найдено никаких рекомендаций касательно
способов идентификации USB устройств.

\clearpage
\subsection{UEFI Shell}
UEFI (Unified Extensible Firmware Interface) — интерфейс между OC и программами, функционирующими с низкоуровневым оборудованием \cite{uefi}.
Во многих современных компьютерах вместо старой системы ввода-выводы BIOS, используется более современный UEFI. Тем не менее, не на всех
вычислительных машинах, на которых установлен UEFI, есть доступ к командной строке. Это легко решается загрузкой с флэш-накопителя, на котором
заранее предустановлен Shell.efi.
\begin{figure}[H]
  \includegraphics[scale=0.5]{images/uefi.png}
  \caption{Дерево устройств в UEFI Shell}
\end{figure}
К сожалению в UEFI Shell доступно ограниченное количество консольных команд, однако с помощью devtree можно получить вывод дерева устройств,
подключенных к компьютеру. Как можно видеть, мы не можем получить удобочитаемое и хотя бы примерно похожее на полученное ранее
представление устройств на шине USB. EUFI Shell использует собственный алгоритм для генерации идентификатора пути USB устройства. В документации
указано, что каждая платформа генерирует свой собственный уникальный путь для USB устройств \cite{key}.
\clearpage
\fi